%LTeX: language=de-DE
\chapter*{Ideen und Materialsammlung}
    Im Zuge der Prüfungsleistung soll sich dieses Dokument mit den chemischen und physikalischen Eigenschaften einerseits
    und den ökonomischen und ökologischen Aspekten andererseits des Kunststoffes Acrylnitril-Butadien-Styrol - kurz ABS -
    auseinandersetzen. Darüber hinaus wird auch der eigene (kennen)Lernprozess mit dem Themenkomplex während des Erstellens
    des Dokumentes reflektiert.\par

    \section*{Wie in der Session besprochen}
    Industrieverband Kunststoffes\par
    Umweltverband Kunststoffe\par
    \textit{Allgemein Industrie anhauen gute Idee}\par\medskip

    Wikipedia nicht zulässig\par
    Institute (Fraunhofer, Max-Plank, etc.) gehen klar\par
    \textit{Allgemein revidierte (peer review) Quellen}\par


    \section*{Allgemeines}
    Farblos/grau
    \begin{figure}[H]
        \begin{framed}
            \textbf{Bild?}
        \end{framed}
    \end{figure}
    CAS 9003-56-9 \cite{en.Wikipedia.2020.ABS}

    \section*{Physikalische Eigenschaften}
    Dichte: \( \SI{1.06-1.08}{\nicefrac{g}{cm^3}} \) \cite{en.Wikipedia.2020.ABS}
    
    Reissdehnung nach DIN 53455: \( \SI{15-30}{\percent} \) \cite{Wikipedia.2020.ABS}

    Lin. Ausdehnungskoeff: \( (60-110) \cdot 10^{-6}{K^{-1}} \) \cite{Wikipedia.2020.ABS} vs. \( \SI{12 \cdot 10^{-5}}{K^{-1}} \) \cite{en.Wikipedia.2020.ABS}

    Spez. Wärmekapazität: \(\SI{1,3}{\nicefrac{kJ}{kg \cdot K}}\) \cite{Wikipedia.2020.ABS}

    Dauergebrauchstemperatur: \( \leq \SI{85-100}{\celsius} \) \cite{Wikipedia.2020.ABS}

    El. Durchschlagsfestigkeit: \( \leq \SI{120}{\nicefrac{kV}{mm}} \) \cite{Wikipedia.2020.ABS}

    UV-beständig \cite{Wikipedia.2020.ABS}

    \textit{Temperaturbeständig}\footnote{wie sehr? unter welchen bedingungen?} \cite{Wikipedia.2020.ABS}

    Beständig gegen Öle und Fette \footnote{welche? warum?} \cite{Wikipedia.2020.ABS}

    \section*{Chemische Eigenschaften}
    \begin{figure}[H]
        \centering
        \includesvg[inkscapelatex=false, width=.7\textwidth]{referenzen/ABS_Monomers_V3}
        \caption[Die drei Monomere]{(Die?) Zur Herstellung notwendigen Monomere \cite{en.Wikipedia.2020.ABS}}
        \label{fig:Basic Monomere}
    \end{figure}

    \section*{Rohstoffe}

    \section*{Herstellung}
    ABS besteht aus den drei Monomeren Acrylnitril, Butadien und Styrol. Die folgenden Abschnitte sollen einen groben
    Überblick zur Synthese der jeweiligen Monomere verschaffen, um darauf aufbauend Verfahren zur Polymerisation aufzuzeigen.
    %
    \subsection*{Acrylnitril}
    Acrylnitril wird durch Addition von HCN
    \subsection*{Butadien}
    \subsection*{Styrol}

    Acrylnitril/Polybutadien/Styrolpfropfpolymere (ABS)
    Herstellung
    \begin{itemize}
        \item Pfropfpolymerisation von Styrol und Acrylnitril auf einen vorgelegten Polybutadienlatex; das erhaltene Pfropfpolymerisat
        wird mit einem getrennt hergestellten SAN-Latex abgemischt, koaguliert und getrocknet. \cite{Domininghaus.1998}
        \item Pfropfpolymer und SAN werden getrennt hergestellt, isoliert und getrocknet, schließlich nach dem Abmischen granuliert. \cite{Domininghaus.1998}
    \end{itemize}
    Quelle wurde 1998 herausgegeben -> höchstwahrscheinlich, dass die Verfahren nicht mehr "die wichtigsten" sind.

    \section*{Einsatzzweck(e)}
    3D-Druck (siehe typische Verarbeitungstemperatur)

    \section*{Entsorgung/Recycling}
    Tja, wie werden wir den Kram wieder los?

    \section*{Vor/Nachteile}