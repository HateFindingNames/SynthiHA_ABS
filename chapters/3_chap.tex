%LTeX: language=de-DE
\chapter{Eigenreflexion und Zusammenfassung}
    Dadurch, dass es derzeit risikobehaftet ist sich mit Printliteratur zu versorgen und die verfügbaren digitalen Medien\footnote{Der HLB.}
    zu dem Thema eher allgemein gehalten sind fiel es schwer, sich ein tieferes Verständnis zu dem Werkstoff anzueignen.
    Sämtliche an diverse Hersteller gerichtete Anfragen – etwa \textsc{BASF}, \textsc{Ineos}, \textsc{Stratasys} – blieben bislang unbeantwortet.
    Obschon unbefriedigend wird die konkrete Rezeptur, zu der auch die Wahl der Additive gehört, Geschäftsgeheimnis sein\footnote{Es lässt sich natürlich nicht völlig ausschließen, dass meine Mails dem menschlichen Filter der Bürokräfte erlagen.}.
    
    \medskip
    Die ungeahnte Komplexität des Themas kann schnell überwältigen. Durch die Möglichkeit mittels unterschiedlicher Mischverhältnisse
    ein quasi-kontinuierliches Spektrum verschiedener Derivate zu erzeugen, die jeweils mit ihren speziellen Eigenschaften
    und Einsatzgebieten aufwarten, ist es schwer eine belastbare Aussage über die Eigenschaften \textbf{des} ABS zu machen.
    Als valide Methode bot sich an, \(n\) Datenblätter aus \(m\) Herstellern/Produzenten zu vergleichen und daraus Minimum,
    Maximum und Median zu ermitteln. Der zeitliche Imperativ gab jedoch Anlass hiervon wieder Abstand zu nehmen um
    alternativ Materialdatenbanken zu rate zu ziehen. Doch auch hier gibt es eine überwältigende Vielfalt teils widersprüchlicher Informationen.
    Ein Vergleich der Angaben bezüglich der maximalen dauerhaften Gebrauchstemperatur zwischen \textsc{Dielectric Manufacturing} \cite{ABS.Datasheet.dielectricmfg.20190227}
    und \textsc{MatWEB} \cite{materialdatenbank.ABS.matweb.com.20210210} etwa liefert ein Delta von \(\approx \SI{33}{\celsius}\).
    Modifikationen durch Wahl verschiedener Additive ergänzen die jeweiligen Portfolio um weitere, spezialisiertere Varianten.
    So ist es auch hier bisweilen nicht völlig transparent, ob es sich bei einem unter einem Handelsnamen erhältlichen Material
    um lediglich abgemischtes oder durch Additive bzw. Blends weiterentwickeltes ABS handelt.
    
    \medskip
    Es ist mir bis dato noch nicht völlig klar, warum die Kunststofffamilie ABS noch derart weite Verbreitung auf dem Markt
    findet während Alternativen relativ problemlos erhältlich und in die bestehende Infrastruktur integrierbar sind.
    Beispielhaft etwa das eng verwandte ASA welches einerseits bereits seit knapp zwei Dekaden bekannt ist und produziert
    wird und andererseits dem ABS in Teilen deutlich überlegen ist.

    Nichtsdestoweniger scheinen die Vorteile des ABS gerade für Produkte \enquote{in Menschennähe} auf der Hand zu liegen.
    Durch breite Verfügbarkeit bei niedrigen Materialkosten und günstigen Verarbeitungseigenschaften während unaufwändig hohe
    Oberflächengüten gewährleistet sind, drängt sich das Material geradezu auf. Lebensmittelverträglichkeit out-of-the-box
    leistet hier sein Übriges.