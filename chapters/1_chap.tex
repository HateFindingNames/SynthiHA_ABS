%LTeX: language=de-DE
\chapter{Steckbrief}
\section{Einordnung}
    Acrylnitril-Butadien-Styrol oder kurz ABS ist ein Terpolymer, dass zur Gruppe der (amorphen) Thermoplasten gehört. Aufgrund
    der chemischen Struktur ist eine weitere Unterordnung in die \textit{"hochschlagfesten Styrol-Copolymere"}\cite{Eyerer.2020.Polymer.Engineering.1}
    möglich.\par
    Weitere bekannte Vertreter der amorphen Thermoplaste sind das Polyvinylchlorid (PVC), Polystyrol (PS), Polycarbonat (PC)
    oder Polymethylmethacrylat (PMMA). Ihnen allen gemein ist eine gute bis sehr gute optische Transmisivität bedingt durch ihren amorphen Aufbau.
    %
\section{Varianten und Vergleich}
    
    \begin{figure}[H]
        \centering
        \includesvg[inkscapelatex=false, width=\textwidth]{RadarCharts/mechanical/mechanical.svg}
        \caption{Mechanisches Profil.}
        \label{fig:pc mechanical profile}
    \end{figure}
    %
    \begin{figure}[H]
        \centering
        \includesvg[inkscapelatex=false, width=\textwidth]{RadarCharts/thermal/thermal.svg}
        \caption{Thermisches Profil.}
        \label{fig:pc thermal profile}
    \end{figure}
    %
    \begin{figure}[H]
        \centering
        \includesvg[inkscapelatex=false, width=\textwidth]{RadarCharts/chemical/chemical.svg}
        \caption{Chemisches Profil.}
        \label{fig:pc chemical profile}
    \end{figure}
    Noch etwas text

\section{Geschichte}
    Erstmals aufgetaucht 1940 \cite{PlasticsInsight.20201005}
\section{Strukturformel}
\section{Preis}
\section{Herstellung}
    ABS besteht aus den drei Monomeren Acrylnitril, Butadien und Styrol. Die folgenden Abschnitte sollen einen groben
    Überblick zur Synthese der jeweiligen Monomere verschaffen, um darauf aufbauend Verfahren zur Polymerisation aufzuzeigen.
    %
    \section{Acrylnitril}
        Eine Möglichkeit der Synthese von Acrylnitril, die technisch heute noch umgesetzt wird, ist das nach dem gleichnamigen
        Unternehmen benannte \textsc{Sohio}-Verfahren.\par
        Die Ausgangsstoffe sind hier Propen, (Luft-)Sauerstoff und Ammoniak. In Gegenwart eines Katalysators wird der Wasserstoff
        des einfach gebundenen Kohlenstoffatoms abgespalten und der Stickstoff des Ammoniaks lagert sich an. Die Edukte der
        stark exothermen Reaktion sind das gewünschte Acrylnitril und Wasser.\cite{sohio-verfahren.acrylnitril.synthese.Wikipedia.2020b}
    \section{Butadien}
        Das Monomer 1,3-Butadien findet in polymerisierter Form (Polybutadien) überwiegend in der Autoreifenindustrie als synthetischer Gummi
        Anwendung. Es ist ein Nebenprodukt bei der Aufspaltung längerkettiger Kohlenwasserstoffe meist fossilen Ursprungs.
    \section{Styrol}

    \section{Polymerisation}
        \begin{itemize}
            \item Pfropfpolymerisation von Styrol und Acrylnitril auf einen vorgelegten Polybutadienlatex; das erhaltene Pfropfpolymerisat
            wird mit einem getrennt hergestellten SAN-Latex abgemischt, koaguliert und getrocknet. \cite{Domininghaus.1998.Kunststoffe.und.ihre.Eigenschaften,Eyerer.2020.Polymer.Engineering.1}
            \item Pfropfpolymer und SAN werden getrennt hergestellt, isoliert und getrocknet, schließlich nach dem Abmischen granuliert. \cite{Domininghaus.1998.Kunststoffe.und.ihre.Eigenschaften,Eyerer.2020.Polymer.Engineering.1}
        \end{itemize}
\section{Anwendung}
        Aus der Kategorie der Kunststoffe ist Acrylnitril-Butadien-Styrol das meist verwendete Material für Produkt und
        Ingenieursanwendungen.\par
        Aufgrund seiner Temperaturbeständigkeit und hohen Schlagfestigkeit findet es breite Anwendung insbesondere in der
        Automobilindustrie zur Fertigung von Interieur, als Gehäuseteile für Elektronikprodukte oder auch Spielzeug. So verdankt etwa der Hersteller
        \textsc{LEGO} seinen Siegeszug der besonderen Eigenschaftenkomposition des Copolymers. Die thermoplastische Komponente
        des in hohen Anteilen vorhandenen Polystyrols macht die Produktion einfach, schnell, günstig und damit geeignet
        für die Massenproduktion. Die durch das 1,3-Butadien verliehene Elastizität sorgt für Formbeständigkeit auch bei
        wiederholter Nutzung der \textsc{LEGO}-Steine.
\section{Eigenschaften}
        Chemische und thermische Standfestigkeit durch Acrylnitril, Thermoplastisch verformbar durch Polystyrol, Zähigkeit
        und Schlagfestigkeit durch 1,3-Butadien.\par
        Mischungsverhältnisse dieser drei Komponenten beeinflussen die jeweils mit ihnen assoziierten Eigenschaften des
        Gesamtmaterials.
\section{Besonderheiten}