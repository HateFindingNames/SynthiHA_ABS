%LTeX: language=de-DE
\chapter{Klausurfragen}
    \begin{itemize}
        \item[\textbf{F}] Welcher Bestandteil des Acrylnitril-Butadien-Styrol verleiht ihm seine Schlagfestigkeit? Warum?
        \item[\textbf{A}] Polybutadien; Flexibilität durch Freiheitsgrade in den unkonjugierten Bindungen zwischen den Kohlenstoffatomen.
        \vspace{4mm}
        \item[\textbf{F}] Nenne die gängigste Methode zur Darstellung des ABS. Erkläre (grob) den Vorgang.
        \item[\textbf{A}] Pfropfpolymerisation; Auf eine bereits vorhandene Polybutadienkette werden die übrigen monomeren Bestandteile
                            Styrol und Acrylnitril aufgepfropft. (Bonus: das erhaltene Polymer kann zur weiteren Justierung der Endeigenschaften
                            mit SAN abgemischt werden.) 
        \vspace{4mm}
        \item[\textbf{F}] Welcher Teil der Polymerkette ist das \enquote{schwache Glied} bezüglich Photooxidation?
        \item[\textbf{A}] Doppelbindung des Polybutadienrückgrats.
        \vspace{4mm}
        \item[\textbf{F}] Wo liegen die polaren Zentren der Polymerkette? Wie wirken sie sich auf die Gesamteigenschaften des ABS aus?
        \item[\textbf{A}] Am dreifach gebundenen Stickstoff des Acrylnitrils. Er sorgt für Chemikalienbeständigkeit.
        \vspace{4mm}
        \item[\textbf{F}] Welcher monomere Anteil muss erhöht werden, wenn ein hoher Oberflächenglanz gefordert ist? Welche Nachteile können hierbei entstehen?
        \item[\textbf{A}] Styrol; Thermische Beständigkeit wird durch einen höheren Anteil der thermoplastischen Komponente im Gesamtpolymer herabgesetzt.
    \end{itemize}